% -*- mode: LaTeX; TeX-PDF-mode: t; -*- # Tell emacs the file type (for syntax coloring)
\providecommand{\econtexRoot}{}\renewcommand{\econtexRoot}{../../..}
\input{\econtexRoot/Resources/.econtexPaths}
\documentclass{\econtex}
\usepackage{graphicx}
\usepackage{\LaTeXInputs/local}
\usepackage{\packages/econtexSetup}
% \usepackage{\packages/econark}
\usepackage[normalem]{ulem}
\newcommand\redout{\bgroup\markoverwith
  {\textcolor{red}{\rule[.5ex]{2pt}{1pt}}}\ULon}

\pagestyle{plain}
\hypersetup{pageanchor=false} % prevents useless warning messages
\pdfinfoomitdate 1
\usepackage{\packages/pdfsuppressruntime}

\begin{document}
\renewcommand{\thepage}{} % Get rid of page numbers, which don't convert to md or html

% \hfill{\tiny \jobname, \today}
% \vspace{.1in}

\begin{verbatimwrite}{\jobname.title}
  Recommendation Letters Procedures
\end{verbatimwrite}
\centerline{\textbf{{\Large Recommendation Letters Procedures}}}\medskip\medskip

% {\today}\medskip\medskip

\section{Information for Students and Staff}\hypertarget{students-and-staff}{}

The department will send out recommendation letters in these ways:
\begin{enumerate}
\item {\EJM}~ ~ (\EJMLink)
\item \AJO ~ (\AJOLink)
\item \AEA ~ (\AEALink)
\item \Interfolio ~ (\InterfolioLink)
\item ``Pull'' Email (that is, when the employer, after receiving your application, sends an email to your recommenders asking them to upload the letters to a special-purpose site that the employer has set up)
\item ``Push'' Email (that is, when the employer provides an email address to which letters should be sent)
  % \item Snail Mail
\end{enumerate}

All of these methods require that your recommenders send PDFs of their letters 
to~\jmstaffemail.\footnote{A few employers may demand that the PDFs be ``digitally signed.''  For now, we will ignore this demand.  If and when those employers start to make up a critical mass, we will have to figure out how to train the faculty in how to make digitally signed letters.}  However, you should NOT ask them to send their letters until you have done the things that YOU are supposed to do (detailed below).

Note that these options do NOT include a method of getting letters to
any employer that has set up its own database system and wants
recommenders and/or students to register for a username, password,
etc.  In practice, such places will invariably accept letters sent by
email to some department staff person, and that is what we will do.
YOU need to find out the email address of that staff person.  If there
are any employers that absolutely insist that applicants and
recommenders learn to use their own unique systems, letters to those
employers will be completed only AFTER ALL OTHER LETTERS FOR ALL OTHER
STUDENTS have been sent using the preferred methods listed above.
This is vitally important because from past experience we know that
the confusion and delays caused by proprietary systems have the
potential to end up making everyone's letters late, so we can do this
only once that risk has been eliminated.  The only exception to this
rule is the IMF, which has its own procedures that we do follow.

Some employers have a deadline before Nov 1 (a few as early as Oct 
15), but the median employer has deadlines in middle to late November.
Finally, employers with late job postings may appear in the December
JOE.

This means that, in your {\EMW} worksheet, you will need to mark each
employer with an indicator for which `batch' the letters should be
assigned to: 

\medskip

\begin{quote}
\begin{quote}
    \begin{table}[h]
  \begin{tabular}{l|l|l}
\hline    Batch     & Contains                        & You Finish By        \\ \hline
    Early     & ALL due BEFORE Nov 15           & Oct 15               \\
              & AND HALF of Nov 15 deadlines    &                      \\
    Middle    & Remaining Nov 15 deadlines      & Nov 1                \\
    Late      & Post-Nov 15 deadlines           & Nov 15               \\
    Late-Post & New listings as they trickle in & Ongoing after Nov 15 \\
              & Communicate with {\JMStaffName} &                      \\ \hline
  \end{tabular}
\end{table}
\end{quote}
\end{quote}

You will keep track of the employers that correspond to each batch 
by attaching the appropriate label to that employer in a special
column in the spreadsheet.  There should be at 
least a week between the date when you send the spreadsheet and the
earliest application due-date contained therein.  Finally any new job postings
that trickle in after Nov 15 will be handled on a piecemeal basis
as part of the ``Late-Post'' batch.

Broadly, for each iteration your steps are as follows:
\begin{enumerate}
\item Produce your version of \EMtt~(by ``your version'' I mean, of
  course, to rename the template file to, for example,
  \texttt{EmployersThomK-Early.xlsm} if you are Kevin Thom and it is
  your \Early~list) that contains \textit{all} the employers, and \textit{only} the employers, that you are actually applying to in this
  round.  (You probably will have some employers you have thought
  about but have decided not to apply to, or to apply to in later rounds; if you want to preserve that
  information, please copy and paste it to another spreadsheet, and
  remove it from your main~{\EMW}~spreadsheet that you will give to the
  staff.)

  
\item Sort your \EMtt~spreadsheet according to five sort keys:
  \begin{enumerate} 
  \item Date on which the staff are to send the recommendation letters; 
  \item Method of recommendation (\AEA, \AJO, {\EJM}~, \Interfolio, push email (for sorting purposes,
    label it ``email-push''), pull email (label it ``email-pull''), or
    ``other'' with the understanding that anything in ``other'' will be
    sent only after ALL other recommendations for ALL other students are
    sent);
  \item Academic vs.\ Nonacademic;
  \item Domestic vs.\ Foreign;
  \item Employer Name. 
  \end{enumerate} 

  (There should be a macro (\texttt{Ctrl}+\texttt{w}) built into \EMtt~that can be
  executed to accomplish this sort -- see the instructions in
  \EMtt's \texttt{Instructions} tab).

  % \item \redout{After sorting the spreadsheet, execute the macro to ``hide'' the information that \JMStaff~does not need for purposes of tracking which letters have been sent.  (Instructions for how to execute this macro should be on the first page of \EMtt).}

% \item Then send an email message {\EMtt}~to {\jmstaffemail}~AND to {\JMPOEmail}, letting us know that the list has been posted, AND containing a clickable link to the file.  For example, if I were applying in 2020, and this was my ``Early'' list, I'd send a message like:

%   \begin{quote}
%     \begin{texttt}
%       To: \jmstaffemail , \JMPOEmail
      
%       Subject: EmployersCarrollCD.xlsm is posted

%       \indent I have posted my first Employers speadsheet at:

%       (link to the spreadsheet in OneDrive)
      
%     \end{texttt}
%   \end{quote}    

%   (The placement director needs to know where you have applied for a host of reasons, including being prepared for calls that employers might make seeking further info).

  \item \redout{You'll probably want next to immediately ``unhide'' those columns that were hidden, so that you can see them yourself.  Again, there should be a macro that accomplishes this described in the first-page instructions.}


\item \textit{After having sent} their~{\EMW}~spreadsheet, the student
  should follow the steps outlined in detail below for those
  applications that involve either push or pull email.

\item \textit{After taking care of their push and pull email applications}, students should complete their {\EJM}~, \AJO, \AEA, and \Interfolio~applications for this round (see below for details).

\item \textit{After having done all of this}, the student should email their recommenders and ask them to send their letters to {\jmstaffemail}.  Note that your request to recommenders needs to come last because if the letters arrive before the other steps have been taken, the employer probably won't have a file for you and the letters may get lost.  (Though it would be wise to remind your advisors a week or two beforehand that letters from them will be needed soon).

\end{enumerate}

%More detailed procedures are below (see also the \href{\pageurl/RecLetters/#information-for-faculty}{Information For Faculty} section below):

\begin{enumerate}

\item \textbf{\EJM ~ (\EJMLink):}
  \begin{itemize}
  \item For JHU faculty recommenders, {\EJM}~identifies recommenders using
    an official JHU email addresses.  

    \begin{itemize}
    \item Many faculty already have an~\EJM~account; for these
      recommenders, it will be obvious how to make~\EJM~send the
      recommender a request for a letter.

    \item Some faculty members may not yet have an~\EJM~account.  For security reasons, \EJM~now insists that each recommender can have only one account at \EJM, associated with a unique \texttt{@jhu.edu} email address.  Even though the faculty member may never use that email address (instead, for example, using a \texttt{gmail} account for all correspondence), they nevertheless do \textit{ have} a JHU email address.  If they don't know what it is, they can ask \jmstaffemail.  You need to find out from the faculty member what email address they want you to give to \EJM~when it asks for their address.  Once you give~\EJM~that email address, a message will be sent to the email address informing the faculty member that an account has been established in their name at~\EJM.  After logging in, they should designate \jmstaffemail~as a ``proxy'' or ``surrogate'' who can load their letters for them, and so when you request subsequent letters the request should actually go to \jmstaffemail, who will receive a ``pull'' email notification and has to click on the link provided for each recommender, click on the name of the student who required the letter, upload the letter, and select all the employers.

    \end{itemize}

  \item Recommenders who are not Department of Economics full-time faculty and who do NOT have their own
    login ID at \EJM~will need to have an account created for them (the account is created automatically the first time a student identifies the recommender by giving \EJM~the recommender's email address), then they must upload their letters themselves.\footnote{They follow the same steps that the
      staff completes in the bullet point about JHU faculty.  They
      log in, they upload their letter, they indicate that it's for, e.g.,
      \texttt{colleen.carey@jhu.edu}, and then they select all employers and tick the box for all
      future employers.}
    New \EJM~security measures prevent us from uploading letters on behalf of people
    who are not JHU faculty, including recently departed JHU faculty who still have an official JHU email (the letter writer must write from their new institution).  The student must communicate this information to  the recommender.

  \end{itemize}

  
\item \textbf{\AJO ~ (\AJOLink):}
  Students will enter {\JMStaffName}'s name, with the email address \jmstaffemail, on the cover sheet provided by the AJO system. Check the box: ``must check here if the person above will upload letters on behalf of multiple writers'' and enter the actual writers' names in the box provided. After all reference letters for each student are received, {\JMStaffName} will make 1 PDF file of all letters and upload their file to the AJO site. The student will be allowed to see when this upload is complete. This service will be provided for both JHU Economics and external recommenders.

\item \textbf{\AEA ~ (\AEALink):}
  Students who are using \AEA~must ask their JHU Economics Department reference writer to go to \AEARecLink~ and set up a surrogate for their reference letters. The surrogate name is {\JMStaffName}~{\JMStaffNameLast}~and the email address is \jmstaffemail. %This service will be provided for JHU Economics professors only. Recommenders who are not Department of Economics full-time faculty must upload their letters themselves, including recently departed JHU faculty who still have an official JHU email (the letter writer must write from their new institution).

  For details, see the \href{\pageurl/RecLetters#information-for-faculty}{Information For Faculty} section.
  
\hypertarget{students-interfolio}{}
\item \textbf{\Interfolio ~ (\InterfolioLink):}
  Before using the Interfolio System, students must wait for an email from {\JMStaffName} indicating that she has received \textit{all} your letters of recommendation. {\JMStaffName} will tell you if you have multiple letters of recommendation from a specific faculty member as well as the type of recommendation (for example: Generic, Academic, or Non-Academic). \textit{Do not proceed} with the recommendation requests if you don't know the type of recommendation letter you are getting from each specific faculty member.

  When entering the recommendation requests in Interfolio please make sure to:
  \begin{itemize}
  \item Start the recommendation request by going to ``Letters'' in the left menu bar.
  \item Enter the recommender as {\JMStaffName}~{\JMStaffNameLast} and \jmstaffemail~as the e-mail for each recommendation request. You need to do this as many times as the number of recommendation letters you are expecting to have.
  \item For each recommendation request, specify under ``Document Title'' the recommender's actual name (e.g. Dr. Laurence Ball) and the type of the recommendation letter (generic, academic, non-academic. etc.) per {\JMStaffName}'s e-mail. For example, if {\JMStaffName} emailed you that Professor Ball wrote you both academic and non-academic letters of recommendation, one of your document titles would be: \texttt{Dr. Laurence Ball\_non-academic recommendation} and another would be \texttt{Dr. Laurence Ball\_academic recommendation}.
  \item ``Description'' -- leave blank.
  \item ``Message to Recommender'' -- leave as is.
  \item Under ``Recommendation Type'' there are two choices -- General and Specific position. Only choose ``Specific position'' and enter the recommender actual name along with the type of the reference letter (generic, academic, non-academic). For example: \texttt{Dr. Laurence Ball\_non-academic recommendation}.
  \item ''Due date''  - this is irrelevant, just choose a date a few days ahead. Your actual deadlines are already included in the spreadsheets sent to {\JMStaffName}. 
  \end{itemize}

  All Interfolio reference letters will be uploaded to Interfolio directly by {\JMStaffName}, including the ones from external recommenders.

\item \textbf{Pull email}: Some employers have set up systems that allow
  recommenders to upload letters directly themselves.  These employers
  will ask the student to provide the email address of each person who
  is to provide a letter.  The employer then sends an email to each of
  those email addresses, requesting that the letter be uploaded.
  (This is a ``pull'' system because the employer is trying to
  ``pull'' the letter from the recommender.)

  In order for us to keep track of letters in a centralized way, and
  to relieve recommenders of the burden of figuring out how to
  upload their letters, our procedure is as follows.  When the
  employer asks for the email address of a JHU faculty recommender, you should
  always reply with \jmstaffemail~rather than the faculty member's
  actual email.

  If the recommender is not a JHU faculty member but they would like \JMStaff~to handle uploading their letters, you should just use \jmstaffemail~as the recommender's email address.  If the recommender \textit{wants} to handle their letters themselves, then you can give the employer their real email address.  But this is discouraged, because it means that we do not have any way to track whether the letters have been sent or not.

\item \textbf{Push email:} Some employers just provide an email address
  to which letters should be sent.  In order to speed up the process,
  students will provide the staff with as few as possible email groups
  (in general two, for academic and non-academic employers). For push email, we do not 
  make any distinction between JHU faculty recommenders and outside recommenders; \JMStaff~will
  simultaneously send out letters of both kinds of recommenders using the procedure below.  Students will send to \jmstaffemail~ one single email
  with subject ``email lists for recommendation letters for (your
  name)'' whose content will look like this: \small
  \begin{itemize}
  \item \textbf{ GROUP 1} (recommenders: Prof.\ Carroll, etc.)
    \begin{itemize}
    \item \textbf{ subject:} Academic Recommendation Letters Needed for (your name)
      % \item \textbf{ bcc:} CarrollCDJHUEconJobMarket@gmail.com, SecondRecommenderJHUEconJobMarket@gmail.com
    \item \textbf{ Body:} employer1@aaa.aaa; employer2@aaa.aaa; employer3@aaa.aaa; etc.
    \end{itemize}

  \item \textbf{ GROUP 2} (recommenders: Jesus H. Christ, Carl Christ, Adrian Pagan, etc.)
    \begin{itemize}
    \item \textbf{ subject:} Nonacademic Recommendation Letters Needed for (your name)
      % \item \textbf{ bcc:} ChristJHJHUEconJobMarket@gmail.com, ChristCJHUEconJobMarket@gmail.com
    \item \textbf{ Body:} employer1@aaa.aaa; employer2@aaa.aaa; employer3@aaa.aaa; etc.
    \end{itemize}

  \item \textbf{ GROUP 3} (recommenders: James Bond, etc.)
    \begin{itemize}
    \item \textbf{ subject:} Extra super secret special jobs Recommendation Letters for (your name)
      % \item \textbf{ bcc:} BondJamesBondJHUEconJobMarket@gmail.com, ...
    \item \textbf{ Body:} employer1@aaa.aaa; employer2@aaa.aaa; employer3@aaa.aaa; ...
    \end{itemize}
  \end{itemize}

  \textbf{NOTE:} \textcolor{red}{\textit{All email addresses included in the groups must be separated by a semicolon.}}

  \normalsize

  The staff should create one email for each group (copying and pasting
  the ``Subject'' fields from the student's email) and attach
  the appropriate letters (academic or nonacademic).  If error
  messages are returned (for example for a mistyped email address, or
  over quota of the recipient's account, etc.), the staff will forward
  the error message (but not the recommendation letter!) to the student
  in question. The student must determine the reason for the error and
  provide the staff with an alternative email address.

  % \begin{comment}
  % \item \textbf{Snail Mail.}
  %   Students will ask the staff for the
  %   appropriate number of business size envelopes (with the Johns
  %   Hopkins return address), attach the address labels for their
  %   employers (the address labels are created by a special macro in the
  %   spreadsheet -- see
  %   \url{http://www.econ2.jhu.edu/jobmarket/Information/}), and return
  %   them divided into 5 groups (academic-USA, non-academic-USA,
  %   academic-International, non-academic-International) held together by
  %   rubber bands (one envelope per employer). On the first envelope of
  %   each group, students will attach a post-it reporting their name, the
  %   letter type (academic or non-academic), destination (USA or
  %   international), the names of the recommenders and the number of
  %   envelopes. Here is an example of the post-it:

  %   \begin{center}
  %     \begin{tabular}{c}
  %       Student's name
  %       \\ Academic - USA
  %       \\ Prof. Carroll
  %       \\ Prof. Faust
  %       \\ Prof. Jeanne
  %       \\ 35 envelopes
  %     \end{tabular}
  %   \end{center}

  %   Staff can thus pick up a group of letters and immediately see how many
  %   copies of the letters to print, the recommenders' names, and the letter
  %   type (academic or non-academic). Staff can ask students to help
  %   sealing the envelopes and applying postage.  The mail is picked up at
  %   1:30 in the afternoon, so the staff will usually need at least a
  %   couple of days to get your letters out.  Further advice from several
  %   former Hopkins students (successful job seekers!) is available in the
  %   documents labelled \texttt{AdviceMoniker.pdf} in the Resources folder
  %   on the \texttt{JHUJobMarket} page.
  % \end{comment}
\end{enumerate}

\begin{comment}
  In sum, both inside and outside recommenders send their letters to \jmstaffemail.  Details are:

  \begin{quote}
    \begin{itemize}
    \item[EJM] New inside recommenders should set \jmstaffemail~as their proxy (old inside recommenders should already have done this in prior years, and need to do nothing at all beyond sending letters to \jmstaffemail). {\JMStaffName} can only proxy JHU Economics Faculty letters. \textcolor{red}{Recommenders who are not Department of Economics full-time faculty must upload their letters themselves.}
      \textcolor{red}{\item[AJO] New inside recommenders should set \jmstaffemail~as their proxy (old inside recommenders should already have done this in prior years, and need to do nothing at all beyond sending letters to \jmstaffemail). {\JMStaffName} can only proxy JHU Economics Faculty letters.} \textcolor{red}{Recommenders who are not Department of Economics full-time faculty must upload their letters themselves.}
      \textcolor{red}{\item[AEA/JOE] All inside recommenders should set \jmstaffemail~as their proxy. The faculty need to set {\JMStaffName} \JMStaffNameLast as their proxy each year, as this system does not carry their proxies over from year to year. {\JMStaffName} can only proxy JHU Economics Faculty letters.} \textcolor{red}{Recommenders who are not Department of Economics full-time faculty must upload their letters themselves.}
      \textcolor{red}{\item[Interfolio]  Inside and outside recommenders follow the same procedures, which require the student to send carefully-constructed emails to \jmstaffemail~as described above.}
    \item[Push Email]  Inside and outside recommenders follow the same procedures, which require the student to send carefully-constructed emails to \jmstaffemail~as described above.
    \item[Pull Email]  For inside recommenders, student requests employer send email to \jmstaffemail.  For outside recommenders, student requests employer send email to recommender's outside email address.
      % \item[Snail Mail]  Inside and outside recommenders follow same procedures, described above.
    \end{itemize}

  \end{quote}

\end{comment}

\section{Information for Staff}\ifdvi\hypertarget{StaffInfo}{(StaffInfo)}\fi

Most of what the staff will do for recommendation letters is implicit in the previous section.  To
summarize:
\begin{enumerate}
\item The \EMtt~spreadsheet from a student should be accessible to you automatically.  When you receive it, insert a new Column A and save the spreadsheet to the students electronic folder located on the (X) drive or Stella as it is named. This new column will be used to mark off each employer once their letters are sent to them.
\item \textit{After} you receive that spreadsheet, you should start receiving requests for letters through~\jmstaffemail.  
\item Some naughty students may request letters via~\EJM~or the push or pull email procedures before they have created their spreadsheet; if you receive such requests before having received the student's spreadsheet, please send them a message asking them to IMMEDIATELY construct their \EMtt spreadsheet (and put it in the right place).
\item Whenever you fulfill a request for letters through~\jmstaffemail, keep track in the student's spreadsheet.
\item If a deadline is looming and you do NOT have a checkmark on the student's spreadsheet, there may be a problem of some kind: The employer mistyped the email address for \jmstaffemail, the student made a mistake somewhere, etc.  As the big deadlines approach, please look at the spreadsheets and see whether there are letters that should have been sent but have not been.
  % \item If different staff persons are handling different categories of letters (like, one person is handling snail mail letters and another is handling the rest), then each person responsible for a particular category of letters should print their own copy of the spreadsheet and keep track of the letters they personally have sent.
\end{enumerate}
\hypertarget{information-for-faculty}{}

\section{Information for Faculty}%\ifdvi\hypertarget{FacInfo}{(FacInfo)}\fi

The list of ways in which we are prepared to send letters is available {\href{\pageurl/RecLetters/#students-and-staff}{here.}}

You do not need to master all these. Instead, you will send a PDF of your letter to {\JMStaffName}, who will take care of the rest.

However, if you are not already registered with the various authorities, there are a few steps you will need to take, adumbrated below:

\begin{enumerate}

\item \textbf{\AEAref}:
  You presumably already have an~{\AEAref}~account tied to your JHU email address.  (If not, you will need one).

  You will need to go to the \href{\AEAurl}{AEA website}, login,  and click under ``Settings'' (or maybe ``Preferences'') to get to the place where you can register a surrogate, which should of course be \jmstaffemail.

  \hypertarget{let-jmstaff-fulfill-letters}{}
  The job candidate will select you as a writer using the email address under which you are registered on \AEALink.  When a candidate has selected you, the AEA will send you an email informing you of the fact.  If you have already set {\JMStaffName} as your surrogate, you do not need to do anything (except send the letter to the surrogate when ready).


  \begin{quote}
    On both {\EJM} and {\AEAref} there is an option to automatically fulfill letters whenever an employer requests them. Please do NOT do this -- leave it to {\JMStaffName}.
    \begin{itemize}
    \item {\JMStaffName} is designated as the `single source of truth' responsible for knowing which letters have been sent
    \item This is difficult if letters are being fulfilled in ways unknown to {\JMStaffName}
    \item {\JMStaffName} has acquired the expertise to navigate the various systems
      \begin{itemize}
        \item It would be a substantial collective waste of faculty time to ask each faculty member to do this
        \end{itemize}
      \end{itemize}
  \end{quote}
  
\item \textbf{\EJM:} Each JHU faculty member sending a recommendation letter will need to have an account at \EJM ~ (\EJMLink). Your account is first established by the STUDENT who first requests a letter from you. That student must provide to \EJM~a valid \texttt{@jhu.edu} email address for you.  Personal email addresses like those from \texttt{gmail} or \texttt{yahoo} are now prohibited by \EJM~ for security reasons. 

  \medskip
  If you are not yet registered on~\EJM~and a student provides \EJM~with a valid JHU email address for you inside \EJM, that will become your unique~\EJM~email address.

  WARNING: You probably have 4-8 valid email addresses: \texttt{[yourname]@jhu.edu, [variant-with-a-number]@jh.edu, [your-name]@johnshopkins.edu, ...}.  You want to have ONE address known to~{\EJM}~(otherwise, you may have to log in for each email address multiple times).  Even if you have written EJM leters before, a student might try to register you under another address.  If so, I STRONGLY advise you to push back and ask the student to use whatever is your already-registered EJM address.  (You can check whether you are registered by going to {\EJM}~and trying to log in using your best-guess email address.)

  Once you have been registered for \EJM~and received a confirmation email to your JHU account, you can designate a ``proxy'' to handle the actual work of uploading your letters.  We (the Job Market Collective Front) insist that you do things this way, because that provides us (the Front) with a centralized way of keeping track of where the process is.  To repeat, you must NOT upload your letters yourself; you MUST designate a proxy.

  The proxy is {\JMStaffName} \JMStaffNameLast; the email address to use for the proxy is
  \jmstaffemail.  The good thing about using a proxy is that you (as a faculty member)
  need only to send your letters out once, to \jmstaffemail, and everything 
  else is handled for you after that. %This service will be provided for JHU Economics professors only.

  \begin{comment}
    \textcolor{red}{
    \item \textbf{\AJO:}
      Each JHU faculty member must register at \AJOLink~ and enter {\JMStaffName} \JMStaffNameLast as their proxy, using the
      \jmstaffemail~ email address. To do this, log in to your account and click on the `proxy' link near the bottom. This only needs to be done once and the proxy works for all applicants the faculty may have on the job market that year.}
  \end{comment}

\item \textbf{\AEA:}
  Each faculty member must register at \AEARecLink. After you create an account, you will have to designate {\JMStaffName} as a surrogate to manage your reference letters. The surrogate name is {\JMStaffName}{\JMStaffNameLast} and the email address is \jmstaffemail. The faculty need to set {\JMStaffName}~{\JMStaffNameLast} as their proxy each year, as this system does not carry their proxies over from year to year. %This service will be provided for JHU professors only (because any particular faculty member is allowed to have only one proxy, and a departed faculty member will likely have students at their new institution for whom).

\item {\textbf{\AJO} and \Interfolio:}
  Each faculty member should email their reference letters to {\JMStaffName} at \jmstaffemail. Please follow the file naming convention outlined below. This service will be provided for both JHU Economics and external recommenders.

\end{enumerate}

N.B.:  When you send your letters to \jmstaffemail, please use self-explanatory names, like
\texttt{CarrollCD-For-WhiteMN-Academic.pdf}, or
\texttt{CarrollCD-For-WhiteMN-NonAcademic.pdf}, or, if the same letter
is to be sent both to academic and to nonacademic employers,
\texttt{CarrollCD-For-WhiteMN-Generic.pdf}.

\end{document}
