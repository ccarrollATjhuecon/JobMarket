% -*- mode: LaTeX; TeX-PDF-mode: t; -*- # Tell emacs the file type (for syntax coloring)
%\newcommand{\econtexRoot}{..}\input{\econtexRoot/.econtexPaths.tex}\documentclass{\classes/econtex}
\providecommand{\econtexRoot}{}\renewcommand{\econtexRoot}{../../..}
\input{\econtexRoot/Resources/.econtexPaths}
\documentclass{\classes/econtex}

\usepackage{\LaTeXInputs/local}
\usepackage{\packages/econtexSetup}

\provideboolean{MyNotes}\setboolean{MyNotes}{false}
\opt{private}{\provideboolean{MyNotes}\setboolean{MyNotes}{true}} % Whether to show marginalia; useful for initial meeting with students

\pagestyle{plain}
\begin{document}
\hfill{\tiny \jobname, \today} \vspace{0.1in}

\begin{verbatimwrite}{\jobname.title}
  Steps in the Job Market Process
\end{verbatimwrite}

\section*{\Large Steps in the Job Market Process}\medskip\medskip

\begin{enumerate}
\item Familiarize yourself with this document, \timet, and \faq; in all communications I will expect you
  to have thoroughly digested the information contained in all of the resources
  I have provided.

\item In your initial email to the JMPO and JMCC (see \timet~for timing; see \ntn~for definitions of who the JMPO and JMCC are), indicate what your \Moniker~is (don't forget the middle initial!  see \Notation), your main advisor, second advisor, job paper title,
  year in the JHU program, and assessment of likelihood you will
  actually be on the market (for students before 6th year).

  \hypertarget{jhueconpeople-data}{}
\item \ifthenelse{\boolean{MyNotes}}{\marginpar{\tiny Ask computer coordinator to help a student set up a listproc for discussions; secret from JMPO.}}{} In early September, a login entry will be created for you using your \Moniker~in the database that I have set up to keep track of job market candiates, \db~ (its full url is at the end of this document).  Don't be worried about the information you put in this database; it is mainly for internal purposes and we will not (for example) put your phone number on the internet.  But we do \textit{need} your phone number so that you can be contacted quickly if necessary (in response to an inquiry from an employer, e.g.).

  \hypertarget{write-and-post-memo}{}
\item \ifthenelse{\boolean{MyNotes}}{\marginpar{\tiny Ask computer coordinator to help a student set up a listproc for discussions; secret from JMPO.}}{} In mid-September, you need to write a short memo about yourself for me and your advisor(s) that includes most of the facts you have entered into the database, as well as crucial extra content.  In this memo, \textit{briefly} describe your job market paper topic, sketch the rest of your dissertation, mention any other research, and say something about teaching interests.  Give a tentative title for your job market paper and dissertation (this can change later, in the database).  Say what kinds of jobs you are most interested in (e.g.\ university vs.\ teaching college vs.\ non-academic, large vs.\ small school) and any geographic preferences or restrictions.  Note any special selling points (e.g.\ good TA ratings), or special connections that could result in a job offer outside the usual channels.  Include both primary and `backup' contact information so we can always find you quickly.  Also include visa status information.  \textit{This document must not be more than 2 pages.}  Call this document, e.g., \texttt{MemoCarrollCD.pdf}, post it at the root of your \texttt{job market directory} (which should correspond to your moniker; see \ntn), email it to the JMPO and your dissertation advisers, and you're done.  \ifthenelse{\boolean{MyNotes}}{\marginpar{\tiny Think about having Nina create a place where they can deposit these without my intervention.}}{}

\item Note that you do not want to make an unfavorable impression on
  the job market coordinator by being someone I have to hassle to get
  you to do things like this job market memo.  Please do these things
  on time without my having to pester you.

  \hypertarget{Pester}{}
\item Note that you \textit{may} need to pester \textit{me} or your advisor to do things
  that we are supposed to be doing.  Don't be shy about doing this; for pestering
  your advisor, you can blame me and the timetables I have posted (of which all the
  faculty are aware).  If you need to pester me, you can remind me of my own
  schedules and that I have encouraged you to pester me.

  \hypertarget{Schedule-Job-Talk}{}
\item Schedule a practice job talk for a date sometime September in one of the department's workshops (if it is impossible to schedule a date before the middle of October, then you can take one later; but it is \textit{strongly} preferred to do so by mid-October).  (Your presentation must be prepared using the \textit{Beamer} package for {\LaTeX}; or, if you are a computer geek, maybe you can use the Jupyter slides system...)  Make sure to confirm that your advisors can attend on the date of your workshop.  In advance of the workshop, recruit one of your fellow grad students to take detailed notes on any questions or discussions that arise during your seminar.  Also, arrange to make sure that someone is in charge of making a video recording of your pitiful performance using the department's digital videocam (the JMCC will explain to all of you how this works); seeing yourself on camera will prove to be the most effective way to encourage you to correct the cringe-inducing defects in your presentation style.
  
  \hypertarget{Get-AEA-Membership}{}
\item Get an AEA student membership.  Reliable sources tell us that
  the registration process allows you to indicate areas of interest
  and to somehow indicate that you may be on the job market this year.
  Some recruiters may actually use this information!
  
  \hypertarget{Register-For-AEA}{}
\item Register to attend the AEA meetings, and try to get a room in
  the hotel where most interviews will take place (the `headquarters'
  hotel).  Don't wait until the last minute, because late
  registrations do not get included in the directory of who is staying
  where, so employers may not be able to find you.  Plan to arrive no
  later than midday or early afternoon on the day before the first
  sessions; in some cases employers might ask to interview you on the
  afternoon of that day.
  
  \hypertarget{CV}{}
\item Produce a CV (3 pages or less-\textit{no exceptions}) and a
  dissertation abstract (using the provided template and following the
  examples of previous generations of students) by \textit{early} October.  The
  abstract must be approved by your advisor (and this is a good time
  to discuss with the advisor what you expect will be the contents of
  the entire dissertation).  The abstract must fit on one (1) page
  (that is, it must be fewer than two pages, which is to say that the
  number of pages cannot exceed $e^{-\iota \pi}$); it must use at
  least an 11 point font, and have margins of at least 1.2 inches.  Title at
  top, followed by name, followed by text. \textit{Use the templates
    provided.}  Post both the PDF and the \texttt{.tex} documents in
  your root \texttt{job market directory}, so others can learn from your pitiful
  \texttt{.tex} efforts and can be slightly less pitiful themselves.

  \hypertarget{Post-To-Server}{}
\item Once your CV, abstract, and job market paper are ready, it will
  be \textit{your} responsibility to post them into the folder that
  the department will create under your \Moniker.
  Instructions can be found at the \texttt{JMCC Help} web page listed
  below, under the header \texttt{JobMarketComputerHelp}.  To see what
  the final page will look like, go to
  \url{http://www.econ2.jhu.edu/jobmarket/2016/}.  (The page
  \url{http://www.econ2.jhu.edu/computing-resources/}
  may also contain answers to technical questions.)
  
  \hypertarget{Template}{}
\item In order for your job market webpages to be created in a timely fashion, 
  fill out the \texttt{template.txt} file in your server folder promptly.
  Your web pages should contain only the most essential information 
  pertaining to your research, teaching, awards, and references. 
  The details are supposed to be in your CV.
  As an example, please look at the following pages:\\
  \indent {\small \url{http://www.econ2.jhu.edu/jobmarket/2012/CareyCM/}}\\
  \indent {\small \url{http://www.econ2.jhu.edu/jobmarket/2007/ThomKE/}}
  

  \hypertarget{Produce-Job-Paper}{}
\item Produce a finished job market paper by October 1, and send a copy to your advisor and anyone who you will be asking to write letters for you, and post it in your folder on the server.  It should be complete, polished, well-written, and nicely formatted.  It should look like something ready to send to a journal, not a work in progress.  It must be written using {\LaTeX} -- NOT Word, NOT Scientific Word -- {\LaTeX}.  (It's fine to have an additional paper or two ready, but quality is \textit{much} more important than quantity; if you have to make a choice between completing a second or third paper, and polishing the job market paper, polish.)
  
  \hypertarget{\EM}{}
\item Confer with your principal advisor to construct your~{\EMtt} lists (see \hyperlink{below}{below})

  
  \hypertarget{Rec-Letters}{}
\item After you have obtained your advisors' approval for your list of
  places you plan to apply to (and for your job market paper), you
  will follow the procedures in the \\
  ~~~~~~\href{\bloburl/RecLetters}{RecLetters} \\
  directory to get recommendation letters sent on your behalf.

\item \ifdvi\hypertarget{Signal}{\textbf{Signal}~}\fi In the latter half of November, the AEA job registration process 
  allows you to designate no more than two employers to whom you want
  to ``signal'' special interest.  (See the url at the end of this
  document).  These signals, in some cases, actually have value because 
  each person has only two of them.  Though you should not obsess about 
  where to send your ``signals'' you should give it some thought -- if you
  are a plausible candidate for a job at the institution in question, they
  will definitely pay more attention to your application if they receive your 
  ``signal.''  It will single out your application from the 400 others they will have received.
  They might receive only 10 ``signals'' so they can afford to pay close attention
  to those applications.

  You will have the opportunity to send a message -- less than 500 words -- to each signalee.  This should be a carefully crafted message uniquely targeted at explaining why the employer is attractive to you, and why you might be attractive to them.

  
  \hypertarget{Mock-Interviews}{}
\item In early December, we will schedule mock interviews in which
  Hopkins faculty pretend to be interviewers at the AEA meetings and
  grill you as you will be grilled there.

  \begin{quote}
    \href{\pageurl/Steps#Prep-For-Interviews}{\texttt{Prep-For-Interviews}}
  \end{quote}
  
  \hypertarget{Be-Reachable}{}
\item Between early December and late December, employers who are
  potentially interested in you will contact you to schedule an
  interview at the AEA meetings.  It is vital that during this period
  (even over the holidays) you are in touch with your email and phone
  messages.  If you will be away, learn how to retrieve voicemail
  messages from your answering machine before you go!  (Or list as
  your contact phone number a cellphone number that will work wherever
  you may go; note that you CANNOT expect employers to hunt you down -
  the phone number you provide to them should succeed in contacting
  you with 100 percent probability).
  
  \hypertarget{Prep-For-Interviews}{} % \ifdvi\hypertarget{InterviewPrep}{}\fi
\item Prepare for the interviews; several students have provided
  detailed advice, which I have posted in \href{\treeurl/Resources}{\texttt{Resources}}.  In particular, the document
  \texttt{Practice.Interviews.Student.Preparation.doc} provides an
  excellent overview and wise guidance. One further requirement is to be
  able to articulate an explicit plan for when your dissertation will
  be completed and what it will consist of; many employers, in
  interviews, will ask this question, and you need to have an answer
  (approved by your advisor) ready at hand.

  Below is a reminder of what is expected in your ASSA interviews (and which therefore you should prepare for in your mock interviews)
  \begin{quote}
    \begin{itemize}

    \item  Develop a clear, SHORT (30 seconds) description of what your job market paper and dissertation are about. Then have another 2 minutes worth that you can say if you are allowed to get through the 30 seconds without interruption.
    \item  Write out your key points as bullet points, and commit them to memory. Be prepared to be interrupted and still be able to recover and return to say the key things you wanted to say.
      \begin{itemize}
      \item If you can't explain what you have done in a short, memorable way, you are doomed. DOOMED!
      \end{itemize}
    \item  Be confident but not arrogant. Don't look at the floor or ceiling, look at the interviewers. And be sure not to look only at one of them (even if that one is famous) - that insults the other one.
    \item  Be clear about what your contribution is relative to the existing literature. A new theoretical result or approach? A new empirical approach? New data? Better data? Something else?
    \end{itemize}
  \end{quote}
  
  \hypertarget{Job-Talks-Scheduled}{}
\item Beginning in late January, and extending to late March,
  employers who are still interested in you will call to schedule a
  job talk.

  \hypertarget{How-Long-Will-It-Be}{}
\item After any given job talk, an employer can wait anywhere from a
  day to two months before getting back in touch with you.  But it is
  socially acceptable for you to call or email to ask for an update on
  your status.
  
  \hypertarget{Dont-Despair}{}
\item If you don't have a job by late March, don't despair.  In the
  end, experience shows that eventually everyone finds a job!

  \hypertarget{Report-Outcome}{}
\item When your final job status is resolved (that is, when you accept 
  a job offer), email me, your advisors, and the JMCC to make sure we have the info.
  Also, email the other employers who might still be considering you so they will be able to move on to other candidates.
  

\end{enumerate}

\section*{Employers Lists}\hypertarget{employers-lists}{}

One activity that does not fit neatly into the chronological order of steps above above is the process of developing the list of places you want to apply to.

You need to keep track of these using a copy of the \Employers spreadsheet that you will rename to incorporate your \Moniker.

You should start this process even before your advisor gives you preliminary approval to be on the market.  Periodically check Job Openings for Economists (\JOE) (available from the AEA) for job listings; many jobs are already posted as of September 1, and most of the jobs that will be available on JOE will be posted by October 1.  You might also check listings from the National Association of Business Economists.  If you are interested in overseas jobs, look at the European JOE and the ads in the \textit{Economist}.  (If you discover job listings that may be of interest to your classmates, please let them know).

You will ultimately need to generate form-letters to the employers you plan to apply to, and will need various kinds of information about them (like email addresses and phone numbers) as you work your way through the process.  Over the years, we have developed a standardized Excel spreadsheet template that contains slots for all the information you will need to collect, available in the \Templates~folder as \texttt{EmployersMoniker.xlsx}.  You \textit{MUST} use (a renamed version of) this Excel spreadsheet template when you send out your applications (because you want to use the linked form-letter-generating software); therefore, to avoid confusing yourself by having multiple lists, you should start your process by downloading the spreadsheet \Employers from the \Templates~directory.  Rename it to, e.g., \texttt{EmployersCarrollCD.xlsm}, erase the information that is in it (unless you want to apply to some of the example employers included in the template file -- in which case, you still need to verify that they are hiring this year and the contact information has remained the same).  \textit{DO NOT} rearrange or rename the columns in this template -- in the end, the JMPO will need to merge all of the students' templates in order to have a manageable idea of who has applied where, and you do not want to annoy the JMPO at this crucial time by making extra work in the form of figuring out what you have renamed columns to and where you have moved them to.  (You are welcome to add extra columns beyond those in the template, and hide those that you don't need to use; but do NOT rename or reorder them!).

By early October you should have developed a list of most of the places to which you wish to apply.  At that point, make an appointment with your advisor(s) to discuss the list.  When you and your advisor(s) have agreed on the list, you will need to give copies of the list to the people who are writing reference letters.  See \recLet~for procedures. For academic employers, some will ask for evidence of teaching skills; here, there is no clear-cut standard for what they expect to receive, but student evaluations of your performance (including a few choice quotes from the student evaluation forms) are a good idea.  Do NOT send more than a page or two of info on this subject - nobody will read more than this, so distill everything down to the essence.

Err on the side of too many applications rather than too few.  But
don't waste everyone's time by applying to places that you would turn
down under all possible circumstances.
\ifthenelse{\boolean{MyNotes}}{\marginpar{\tiny Emphasize the point
    that they may not apply to places where they would not accept if
    it were the only offer.  If you have an offer and decline it, then
    you will get no further assistance from the department.}}{}

One more point: You should have a regular computer backup plan
(ideally, an automatic backup every night; at a minimum, once a week)
for ALL the materials that are critical to your thesis research,
job market paper, and application process.  Without a backup, you are
at risk of losing months worth of work or more, and in a worst case
scenario it could mean you will have to withdraw from the job market.
In 2003 one student's laptop crashed right in the middle of the job
market.  Fortunately the student had a week-old backup of the key
files; but let that be a warning.

\hypertarget{best-shot}{}
Throughout the process you should keep track (on the spreadsheet) of your set of potential ``best shot'' employers, along with the rationale for why they are in your best shot list.

\begin{enumerate}
\item[Egg] \texttt{EmployersMonikerEarly.xlsm} first draft (mostly empty)
  \begin{itemize}
  \item In the unusual case where you are applying to jobs with a deadline before Sep 1, you should create the spreadsheet from the beginning
  \end{itemize}
\item[Oct-Wk2] List incorporating feedback from advisor(s)
\item[Oct-Wk3] \texttt{EmployersMonikerEarly.xlsm} on OneDrive
  \begin{itemize}
  \item Use the [name] field to mark entries ``ready'' for letters to be sent
  \end{itemize} 
\end{enumerate}


\pagebreak

\together{
  \begin{center}
    {\Large Electronic Job Market Resources}
    \medskip\medskip\medskip
    
    \small
    \begin{tabular}{rl}
      Overall- repo & \href{\repourl}{\repourl} \\
      Overall- html & \href{\pageurl}{\pageurl}
      % \\ Database      & \url{http://jhuEconPeople.dynalias.org}
      \\ Computer Help      & \JMPHLink
      \\ Cover Letter & \href{\bloburl/Templates/CreateLabelandCoverLetterInstruction.md}{\bloburl/Templates/CreateLabelandCoverLetterInstruction.md}
      \\ ASSA meeting & \href{\AEAWeb/conference/}{\AEAWeb/conference/}
      \\ \JOE & \href{\AEAWeb/joe/}{\AEAWeb/joe/}
      \\ \JOE Signal & \href{\AEAWeb/joe/signal/}{\AEAWeb/joe/signal/}
      % \\ Rumors & \url{http://www.econjobrumors.com/}
      % \\ Placements & \url{http://www.econ2.jhu.edu/recent_placements}
      % \\ JMCC Help  & \url{http://www.econ2.jhu.edu/jobmarket/Information/}
      % \\ EconJobMarket & \url{\EJMLink}
      % \\ Job Info & \url{http://www.illinoisskillsmatch.com/}
      % \\ Public Wiki & \url{http://bluwiki.com/go/Econjobmarket}
      % \\ Rankings & \url{https://github.com/ccarrollATjhuecon/JobMarket/Resources/Rankings.doc}
      \\ Advice & \href{\treeurl/Resources}{\treeurl/Resources}
    \end{tabular}
  \end{center}
}
\normalsize

The Advice category is particularly noteworthy, as it contains (among
other things) a variety of documents specifically written by either
JHU students or JHU faculty with their personal advice about various
aspects of the job market process.  For example, the document
\texttt{AdviceMoffittInterviews.doc} contains advice Robert Moffitt
has provided about how to handle job market interviews.

\begin{comment} % Seems defunct, 2012-11-27
  \url{http://www.illinoisskillsmatch.com/} is a site for posting your
  resume and other information for employers to browse.  It may or may
  not be a valuable supplement to the ASSA process.
\end{comment}

One of you (the students on the job market) should create an emailing
list for you to email each other with gossip, tips, useful urls that
you discover, job listings that others might not have seen, etc.  


\end{document}
