\providecommand{\econtexRoot}{}\renewcommand{\econtexRoot}{../../..}

\documentclass[pdflatex,aspectratio=169]{beamer}
\usepackage{\packages/local}


\usepackage{multirow}

\newbool{fullcon}\global\booltrue{fullcon}\boolfalse{fullcon} %full content
\newbool{bundesb}\global\booltrue{bundesb}%\boolfalse{bundesb} %reduced, 20 min presentation
\usepackage{\econark,\econtexShortcuts}

\begin{document}

\begin{frame}
\frametitle{Motivation}
\begin{itemize} 
	\item Various fiscal policies used to fight recessions: payroll tax cuts, stimulus checks, UI extension
	\begin{itemize}
		\itemsep = .25\bigskipamount 
		\item little guidance from traditional RANK models	
		\item different goals: increase output (a `GDP metric') or reduce misery (a `welfare metric')
	\end{itemize}
	\bigskip
	\pause
	\item \textbf{This paper:} Develop a heterogeneous agent (HA) model to study effectiveness of policies in fighting recessions
	\begin{itemize}
		\itemsep = .25\bigskipamount 
		\item Consumers subject to transitory \& permanent income shocks and unemployment risk, heterogeneous in education
		\item Consistent with micro data
		\item Not a HANK model, but aggregate demand multiplier exist during recessions
	\end{itemize}
\end{itemize}
\end{frame}

\end{document}

% _____________ Opening slide _______________________

\begin{document}
\frame

\title[Stimulus]{Welfare and Spending Effects of Consumption Stimulus Policies}
\author{
  Christopher D. Carroll (JHU)
  \and
  Edmund Crawley (FED)
  \and
  Ivan Frankovic (BBK)
  \and
  H{\aa}kon Tretvoll (SSB)
}



\ifbool{bundesb}{% set date for bundesbank
	\date[\today]{Bundesbank in-house research seminar - March 28, 2023  \\ \medskip \medskip \medskip 
	\href{https://econ-ark.org/}{\small Powered By} \\ \includegraphics[width=0.5in]{\econtexRoot/Resources/econ-ark-logo-small.png}}
}{}


\newcommand{\RNum}[1]{\uppercase\expandafter{\romannumeral #1\relax}}

\AtBeginSection[]{
	\begin{frame}
	\vfill
	\centering
	\begin{beamercolorbox}[sep=8pt,center,shadow=true,rounded=true]{title}
		\usebeamerfont{title}\insertsectionhead\par%
	\end{beamercolorbox}
	\vfill
\end{frame}
}

\usepackage[font=small,skip=0pt]{caption}
\usepackage{booktabs}

\begin{document}\bibliographystyle{\econtexBibStyle}

\begin{frame}[plain]
  \titlepage
  
  \footnotesize{Viewpoints and conclusions stated in this paper are the responsibility of the authors alone and do not necessarily reflect the viewpoints of The Federal Reserve Board or The Deutsche Bundesbank.}
\end{frame}


% _____________ 1st section  ____________


	
\begin{frame}
\frametitle{Motivation}
\begin{itemize} 
	\item Various fiscal policies used to fight recessions: payroll tax cuts, stimulus checks, UI extension
	\begin{itemize}
		\itemsep = .25\bigskipamount 
		\item little guidance from traditional RANK models	
		\item different goals: increase output (a `GDP metric') or reduce misery (a `welfare metric')
	\end{itemize}
	\bigskip
	\pause
	\item \textbf{This paper:} Develop a heterogeneous agent (HA) model to study effectiveness of policies in fighting recessions
	\begin{itemize}
		\itemsep = .25\bigskipamount 
		\item Consumers subject to transitory \& permanent income shocks and unemployment risk, heterogeneous in education
		\item Consistent with micro data
		\item Not a HANK model, but aggregate demand multiplier exist during recessions
	\end{itemize}
\end{itemize}
\end{frame}


\begin{frame}
	\frametitle{Model consistent with micro data}
	
	\hypertarget{ConsistentWithMicroData}{}
	
	\begin{columns}
		\begin{column}{0.4\textwidth}
			SCF liquid wealth (Kaplan and Violante, 2014) 
			
			\includegraphics[width=\linewidth]{\FigDir/LorenzPoints.pdf}
			
			Modelling device: \textit{Ex-ante} heterogeneity in discount factors
			
		\end{column}
	
		\pause
		
		\begin{column}{0.4\textwidth}  	
			iMPC from Fagereng, Holm, Natvik (2021)	
			
			\includegraphics[width=\linewidth]{\econtexRoot/Code/HA-Models/Target_AggMPCX_LiquWealth/Figures/AggMPC_LotteryWin.pdf}
			
			Modelling device: `Splurge' in consumption, i.e. exogenously given fraction of income directly consumed
		\end{column}
	\end{columns}
	
	\vspace{0.5cm}
	\hyperlink{ParametrizationStrategy}{\beamerbutton{Parametrization strategy}}
	\hspace{1cm}
	\hyperlink{Parameters}{\beamerbutton{Calibrated parameters}}
	\hspace{1cm}
	\hyperlink{EstimationBetaNabla}{\beamerbutton{Estimated parameters \& fit}}
	
\end{frame}




\begin{frame}
\frametitle{Evaluation of consumption stimulus policies in the US}
\begin{itemize}
\itemsep = .5\bigskipamount 
\item Policies we consider: 
\begin{itemize}
	\itemsep = .25\bigskipamount 
	\item Stimulus check for \$1200 (means-tested)
	\item Extension of unemployment benefits from 0.5 to 1 year
	\item Payroll tax cut by 2\% for 2 years
\end{itemize}
%\item Key features of the policies: 
%\begin{itemize}
%	\itemsep = .25\bigskipamount 
%	\item Targeting 
%	\item Timing of spending (overlap with recession!)
%	\item Scalability 
%\end{itemize}
\bigskip
\item Evaluation criteria: 
\begin{itemize}
	\itemsep = .25\bigskipamount 
	\item Spending multipliers
	\item Welfare (only recession-related welfare impact)
\end{itemize}
\end{itemize}
\end{frame}


\begin{frame}
\frametitle{Preview of results}
\begin{itemize}
\itemsep = \bigskipamount 
\item Welfare measure: Extension of UI benefits is the clear winner 
\begin{itemize}
\itemsep = .25\bigskipamount 
\item Targeted at individuals with high MPCs and high recession-related welfare losses
\item But: higher spending may continue after recession is over 
\end{itemize}
\item Spending multiplier: Stimulus check has the highest multiplier 
\begin{itemize}
\itemsep = .25\bigskipamount 
\item Not well targeted, but increases income immediately 
%\item Also: easy to scale up
\end{itemize}
\item Tax cut
\begin{itemize}
	\itemsep = .25\bigskipamount 
	\item Poorly targeted and much spending likely to occur after end of recession
\end{itemize}
\end{itemize}
\end{frame}

\begin{frame}
\frametitle{Related literature}
\small
\begin{itemize}
\item \textbf{Effects of transitory income shocks}: 
Parker, Souleles, Johnson and McClelland (2013); Broda and Parker (2014); Fagereng, Holm and Natvik (2021); Ganong, Greig, Noel, Sullivan and Vavra (2022)
\item \textbf{HA models consistent with high MPCs}: 
Kaplan and Violante (2014); Auclert, Rognlie and Straub (2018); Carroll, Crawley, Slacalek and White (2020); Kaplan and Violante (2022) 
\item \textbf{State dependent multipliers (ZLB)}: 
Christiano, Eichenbaum and Rebelo (2011); Eggertson (2011); Ramey and Zubairy (2018); Hagedorn, Manovskii and Mitman (2019) 
\item \textbf{Welfare measures in HA models}:
Bhandari, Evans, Golosov and Sargent (2021); D{\'a}vila and Schaab (2022)
\item \textbf{Extended unemployment insurance}:
Ganong, Greig, Noel, Sullivan and Vavra (2022); Kekre (2022) 
%\item \textbf{High MPCs and impatience}: Parker (2017)
\end{itemize}
\normalsize
\end{frame}





\section{Model}

\begin{frame}
\frametitle{Consumer problem}


	\begin{itemize}
		\item Education groups: "Dropout", "Highschool" and "College"
		\item Each group has distribution of subjective discount factors $\beta_i$
		\item Idiosyncratic, stochastic income process $\mathbf{y}_{i,t}$
		\item Estimated splurge factor $\varsigma$: $\mathbf{c}_{sp,i,t} = \varsigma \mathbf{y}_{i,t}$
		\pause
		\item Remaining consumption $c_{opt,i,t}$ is chosen to maximize utility
			\begin{align}
			\sum_{t=0}^{\infty}\beta_i^t (1-D)^t \mathbb{E}_0 u(\mathbf{c}_{opt,i,t}).
			\end{align}
			($D$: end-of-life probability, $u$: stand. CRRA utility func.)	
		\item Budget constraint, given existing market resources $m_{i,t}$ and income state, and a no-borrowing constraint: 
		\begin{align}
		\mathbf{m}_{i,t+1} &= R \underbrace{(\mathbf{m}_{i,t} - \mathbf{c}_{sp,i,t} - \mathbf{c}_{opt,i,t})}_{\geq 0 \text{ (no-borrowing constraint)}} + \mathbf{y}_{i,t+1}
		\end{align}
		($R$: exogenous gross interest rate)
	\end{itemize}



\end{frame}


\begin{frame}
\frametitle{ Income process}

	\begin{itemize}

		\item Income subject to transitory, unempl. and permanent shocks
			\begin{align}
				\mathbf{y}_{i,t} =   \begin{cases}
				\xi_{i,t}\mathbf{p}_{i,t}, & \text{if employed} \\
				0.7 \mathbf{p}_{i,t}, & \text{if unemployed for $\leq$ 2q} \\
				0.5 \mathbf{p}_{i,t}, & \text{if unemployed $\ge$ 2q} 
				\end{cases}
			\end{align}
			($\xi_{i,t}$: trans. shock, $p$: perm. income)
			
		\item "Permanent income":  $\mathbf{p}_{i,t+1} = \underbrace{\psi_{i,t+1}}_{\text{perm. shock}} \underbrace{\Gamma_{e(i)}}_{\text{educ.-specific growth}}\mathbf{p}_{i,t}$ 

		\pause
		\item Emplyoment status is subject to a Markov process
		\begin{itemize}
			\item Unemployment rate education-specific (doubles in recession)
			\item Expected length of unemployment: 1.5q  (4q in recession)
		\end{itemize}
		
		\item Recession is given by an MIT shock; end of recession as a Bernoulli process (avg. length of 6q)
	
	\end{itemize}

\end{frame}



\ifbool{fullcon}{
	\begin{frame}
	\frametitle{Three policies to fight the recession}
	
		\begin{itemize}
		\item Stimulus check
		\begin{itemize}
			\item Everyone receives a check for \$1,200 in q1 of the recession
			\item Check is means-tested: Full check if perm. income $\leq$ \$100k; Falls linearly for higher incomes and zero for those $\geq$ \$150k
		\end{itemize}
	
		\item Extended unemployment benefits
		\begin{itemize}
			\item Unemployment benefits are extended from 2 to 4 q
			\item Extension occurs regardless of whether recession ends
		\end{itemize}
	
		\item Payroll tax cut
		\begin{itemize}
			\item Employees payroll tax rate is reduced such that income rises by 2\% for 8q	
		\end{itemize}
	\end{itemize}
	
		Policies are debt-financed and repayed much later
	\end{frame}
}{}

\begin{frame}
\frametitle{Aggregate demand effects \\ 
	\small (as in Krueger, Mitman and Perri, 2016) \normalsize}
	\begin{itemize}
		\itemsep = .5\bigskipamount 
		\item Baseline: No feedback from aggregate consumption to income
		\item Extension: We allow for aggregate demand effects from consumption on income during the recession
		
		\item The AD effect is given by
			\begin{align}
			AD(C_t) =   \begin{cases}
			\Big(\frac{C_t}{\tilde{C}}\Big)^\kappa, & \text{if in a recession} \\
			1, & \text{otherwise} ,
			\end{cases}
			\end{align}
			where $\tilde{C}$ is the level of consumption in the steady state. 
		
		\item Idiosyncratic income in the extension model is then given by
			\begin{align}
			\mathbf{y}_{AD,i,t} = AD(C_t)\mathbf{y}_{i,t}.
			\end{align}
	\end{itemize}
\end{frame}







\section{Results}


\ifbool{bundesb}{

\begin{frame}
\frametitle{Impulse responses}


	\begin{columns}	
		
		
	\begin{column}{0.5\textwidth} 
		\small 	
		\begin{itemize}
			\item Simulate policies in recessions lasting 1 to 20 q
			\item Construct probability-weighted sum across rec. lengths
		\end{itemize}
	\end{column}	
		
	
	\begin{column}{0.4\textwidth}  
			\footnotesize Stimulus check:
			\includegraphics[width=\linewidth]{\econtexRoot/Code/HA-Models/FromPandemicCode/Figures/recession_Check_relrecession}
				
	\end{column}  	

\end{columns}

\pause
	
	\begin{columns}	
		
		\begin{column}{0.33\textwidth}  	
			\footnotesize Extension of UI benefits:
			\includegraphics[width=1.2\linewidth]{Code/HA-Models/FromPandemicCode/Figures/recession_UI_relrecession} 
		\end{column}
	
		\begin{column}{0.33\textwidth}  
			\footnotesize Payroll tax cut:	
			\includegraphics[width=1.2\linewidth]{Code/HA-Models/FromPandemicCode/Figures/recession_taxcut_relrecession}
		\end{column}
	\end{columns}

\end{frame}
}{}



\ifbool{fullcon}{


	\begin{frame}
		\frametitle{IRFs for stimulus check}
		\centering
		\includegraphics[width=0.6\linewidth]{\econtexRoot/Code/HA-Models/FromPandemicCode/Figures/recession_Check_relrecession}
		\begin{itemize}
			\itemsep = .5\bigskipamount 
			\item W/o AD effects: Q1 income is 5.5\% higher; consumption jumps by 3\% 
			\item With AD effects: Q1 income is 6.5\% higher; consumption elevated for longer time
		\end{itemize}
	\end{frame}

	\begin{frame}
		\frametitle{IRfs for extension of unemployment benefits}
		\centering
		\includegraphics[width=0.6\linewidth]{Code/HA-Models/FromPandemicCode/Figures/recession_UI_relrecession}
		\begin{itemize}
			\itemsep = .5\bigskipamount 
			\item W/o AD effects: quarterly income increases by max 0.7 percent, consumption response shows anticipation of longer duration
			\item With AD effects: extra boost to income by 0.2 percent, consumption stays elevated for longer time
		\end{itemize}
		
	\end{frame}
	
	\begin{frame}
		\frametitle{IRFs for payroll tax cut}
		\centering
		\includegraphics[width=0.6\linewidth]{Code/HA-Models/FromPandemicCode/Figures/recession_taxcut_relrecession}
		
		\begin{itemize}
			\itemsep = .5\bigskipamount 
			\item W/o AD effects: income rises by close to 2 percent; Consumption jumps by 1.5 percent and drops sharply after the income decline.
			\item With AD effects, income rises by 2.5 percent, declines steadily as the recession's likelihood decreases
		\end{itemize}
	\end{frame}

}{}




\begin{frame}
\frametitle{Multipliers when aggregate demand effects are present}


\begin{equation*}
M^P_t = \frac{\text{Net present value of policy-induced consumption up to $t$}}{\text{Net present value of the cost of the policy}}
\end{equation*}

%\begin{figure}[t]
%	\centering
%	\includegraphics[width=0.4\linewidth]{Code/HA-Models/FromPandemicCode/Figures/Cummulative_multipliers}
%	\caption{Cumulative multipliers over time}
%\end{figure}

\begin{table}
	%\footnotesize
	\begin{tabular}{@{}lccc@{}} 
								&Tax Cut   	& UI extension    	& Stimulus check    \\ 
		Multiplier in Q1		&0.05		& 0.25 				& 0.60				\\	
		Long-run Multiplier  	&1.08  		& 1.28  			& 1.34     			\\ 
		Policy expenditure during recession  &57.6\%  & 80.6\%  & 100.0 \%    		\\ 
	\end{tabular}  
\end{table}

\end{frame}


\ifbool{bundesb}{
	\begin{frame}
		\frametitle{Welfare measure construction}
		\hypertarget{WelfareMeasure}{}
		
		Guiding principles
		
		\begin{enumerate}
			\item Each consumer is valued equally by the social planner 
			\item Utility from splurge in the same way as other spending
			\item No social benefit to the policies outside of a recession
		\end{enumerate} 
		
		\vspace{0.6cm}
		
		Simple aggregation of consumer util. only satisfies principle 1 \& 2:
		\begin{align*}
			\mathcal{W}(\text{policy},Rec,AD) =\frac{1}{N}\sum_{i=1}^{N} \sum_{t=0}^{\infty} \beta_S^t u(\mathbf{c}_{it,\text{policy},Rec,AD})
		\end{align*}
		
		\pause
		
		To satisfy principle 3, we calculate
		
		\begin{itemize}
			\item Net welfare: Subtract the welfare cost of financing the policy
			\item Recession-based net welfare: Subtract the net welfare impact of policy outside of recession
		\end{itemize}	
	
		\vspace{0.2cm}
		\hyperlink{WelfareMeasure1}{\beamerbutton{Details on welfare measure}}
		
	\end{frame}

	
}{}




\begin{frame}
	\frametitle{Welfare results}
	\centering 
	\begin{tabular}{@{}lccc@{}} 
		\toprule 
		& Check      & UI    & Tax Cut    \\  \midrule 
		Without AD effects & 0.011  & 0.580  & 0.002     \\ 
		With AD effects & 0.171  & 1.266  & 0.065     \\ 
	\end{tabular}  
	\medskip
	\begin{itemize}
		\itemsep = .75\bigskipamount 
		\item All policies adjusted to the fiscal size of the UI extension
		\item Interpretation: A welfare gain of x $\Leftrightarrow$ social planner is indifferent between 
		\begin{itemize}
			\itemsep = .25\bigskipamount 
			\item the stimulus policy being implemented in response to a recession and 
			\item a permanent increase in the baseline consumption of the total population by x basis points (0.01\% of baseline cons.)
		\end{itemize}
		\item All policies much more effective when mulitplier present
	\end{itemize}
\end{frame}

\begin{frame}
	\frametitle{Conclusion: Comparing the policies}
	\begin{itemize}
		\itemsep = .5\bigskipamount 
		\item Comparison of three consumption stimulus policies in a HA model consistent with data on the distribution of liquid wealth and intertemporal MPCs 
		\item Welfare measure: UI extension is the clear bang-for-the-buck winner 
		\item The stiumulus check is less well targeted, but\ldots 
		\begin{itemize}
			\itemsep = .25\bigskipamount 
			\item is transferred immediately ensuring that money arrives when it is most valuable 
			\item is more easily scaled up to provide more stimulus 
		\end{itemize}
		\item The tax cut is both poorly targeted and may yield substantial spending after the recession is over 
		\item Framework can be used to evaluate other candidate policies 
		
	\end{itemize}
	
\end{frame}



\begin{frame}
	\frametitle{Thank you for your attention!}
	\begin{itemize} 
		\item Access the paper, presentation slides and code at: https://github.com/llorracc/HAFiscal
	\end{itemize}	

		\begin{figure}
			\centering
			\includegraphics[width=0.3\linewidth]{"Presentations/QRCode.png"}
		\end{figure}
	
\end{frame}






\section{Appendix}



\begin{frame}
	\hypertarget{ParametrizationStrategy}{}
	\frametitle{Parametrization --- Strategy}
	\begin{itemize} 
		\itemsep = \bigskipamount 
		\item First: Estimate the splurge factor in a Norwegian version of the economy --- match iMPCs from FHN (2021)
		\item Calibrate a set of parameters that affect all education groups equally 
		\item Calibrate a set of parameters that match features of the different education groups 
		\item Estimate a discount factor distribution for each education group to match within-group distribution of liquid wealth
		\begin{itemize}
			\itemsep = .25\bigskipamount 
			\item $\beta_e$: center of discount factor distribution
			\item $\nabla_e$: spread of discount factor distribution 
			\item Uniform distribution, approximated with 7 different types
		\end{itemize}
	\end{itemize} 
	
	\vspace{1cm}
	\hyperlink{ConsistentWithMicroData}{\beamerbutton{Go back}}
\end{frame}

\begin{frame}
	\frametitle{iMPC from FHN (2021)}
	\centering 
	%	\includegraphics[width=3in]{\FigDir/AggMPC_LotteryWin.pdf}
	\includegraphics[width=3in]{\econtexRoot/Code/HA-Models/Target_AggMPCX_LiquWealth/Figures/AggMPC_LotteryWin.pdf}
	\begin{itemize}
		\itemsep = .5\bigskipamount 
		\item Estimated splurge factor: $\varsigma = 0.31$
		\item Robustness exercise: With $\varsigma = 0$ the fit not as good. 
	\end{itemize}

	\vspace{1cm}
	\hyperlink{ConsistentWithMicroData}{\beamerbutton{Go back}}
\end{frame}

\begin{frame}
	\hypertarget{Parameters}{}
	\frametitle{Parameters --- same for all types}
	\begin{table}
		\small
		\begin{tabular}{lcd{3}} 
			\toprule
			\multicolumn{3}{l}{Parameters that apply to all types} \\ \midrule
			Parameter & Notation & \text{Value} \\ \midrule 
			Risk aversion & $\gamma$ & 2.0 \\ 
			Splurge & $\varsigma$ & 0.307 \\ 
			Survival probability, quarterly & $1-D$ & 0.994 \\
			Risk free interest rate, quarterly (gross) & $R$ & 1.01 \\ 
			Standard deviation of transitory shock & $\sigma_\xi$ & 0.346 \\
			Standard deviation of permanent shock & $\sigma_\psi$ & 0.0548 \\ 
			Unemployment benefits replacement rate (share of PI) & $\rho_b$ & 0.7 \\ 
			Unemployment income w/o benefits (share of PI) & $\rho_{nb}$ & 0.5\\ 
			Avg. duration of unemp. benefits in normal times (quarters) & & 2 \\
			Avg. duration of unemp. spell in normal times (quarters) & & 1.5 \\
			Probability of leaving unemployment & $\pi_{ue}$ & 0.667 \\ 
			Consumption elasticity of aggregate demand effect & $\kappa$ & 0.3 
			\\ \bottomrule 
		\end{tabular}
	\end{table}

	\vspace{1cm}
	\hyperlink{ConsistentWithMicroData}{\beamerbutton{Go back}}
\end{frame}

\begin{frame}
	\frametitle{Parameters --- by education group}
	\label{sli:paramsByEd}
	\begin{table}
		\small
		\begin{tabular}{lccc}
			\toprule 
			\multicolumn{4}{l}{Parameters calibrated for each education group} \\ \midrule
			& Dropout & Highschool & College \\ \midrule
			Percent of population & \phantom{0}9.3 & 52.7 & 38.0 \\ 
			Avg. quarterly PI of ``newborn'' agent (\$1000) & \phantom{0}6.2 & 11.1 & 14.5 \\
			Std. dev. of $\log($PI$)$ of ``newborn'' agent & 0.32 & 0.42 & 0.53 \\
			Avg. quarterly gross growth rate of PI ($\Gamma_e$) & 1.0036 & 1.0045 & 1.0049 \\
			Unemployment rate in normal times (percent) & \phantom{0}8.5 & \phantom{0}4.4 & \phantom{0}2.7 \\ 
			Probability of entering unemployment ($\pi_{eu}^{e}$, percent) & \phantom{0}6.2 & \phantom{0}3.1 & \phantom{0}1.8 
			\\ \bottomrule 
		\end{tabular}
	\end{table}

	\vspace{1cm}
	\hyperlink{ConsistentWithMicroData}{\beamerbutton{Go back}}
\end{frame}


\begin{frame}
	\frametitle{Parameters describing the policies}
	\label{sli:policies}
	\centering 
	\begin{tabular}{lc}
		\toprule 
		\multicolumn{2}{l}{Parameters describing policy experiments} \\ \midrule 
		Parameter & Value \\ \midrule 
		Change in unemployment rates in a recession & $\times 2$ \\ 
		Expected unemployment spell in a recession & 4 quarters \\ 
		Average length of recession & 6 quarters \\ 
		Size of stimulus check & \$1,200 \\ 
		PI threshold for reducing check size & \$100,000 \\ 
		PI threshold for not receiving check & \$150,000 \\ 
		Extended unemployment benefits & 4 quarters \\
		Length of payroll tax cut & 8 quarters \\ 
		Income increase from payroll tax cut & 2 percent \\ 
		Belief (probability) that tax cut is extended & 50 percent 		
		\\ \bottomrule
	\end{tabular}

	\vspace{1cm}
	\hyperlink{ConsistentWithMicroData}{\beamerbutton{Go back}} 
\end{frame}

\begin{frame}
	\frametitle{Estimated parameters}
	\hypertarget{EstimationBetaNabla}{}
	\begin{table}
	\small	
		\begin{tabular}{lccc}
			\multicolumn{4}{l}{Estimated discount factor distributions} \\ 
			& Dropout & Highschool & College \\ \midrule
			$(\beta_e, \nabla_e)$ & (0.694, 0.542) & (0.904, 0.099) & (0.978,0.015) \\
			(Min, max) in approximation & (0.230, 0.995) & (0.819, 0.989) & (0.965, 0.991) \\
			\midrule 
		\end{tabular} 
		\begin{tabular}{lccc}
			\multicolumn{4}{l}{ } \\ \midrule
			\textbf{Estimation targets} & Dropout & Highschool & College \\ \midrule
			Median LW/ quarterly PI (data, percent) & 4.64 & 30.2 & 112.8 \\ 
			Median LW/ quarterly PI (model, percent) & 4.64 & 30.2 & 112.8 %\\
			%         $[20,40,60,80]$ pctiles of Lorenz curve (data) & $[0, 0.01, 0.6, 3.6]$ & $[0.06, 0.6, 3.0, 11.6]$ & $[0.2, 0.9, 3.3, 10.3]$ \\
			%         $[20,40,60,80]$ pctiles of Lorenz curve (model) & $[0.0, 0.0, 0.5, 3.6]$ & $[0.04, 0.9, 3.7, 11.3]$ & $[0.3, 1.5, 4.0, \phantom{0}9.9]$
			\\ \midrule 
		\end{tabular} 
	\footnotesize
		\begin{tabular}{lcccc}
			\multicolumn{5}{l}{ } \\ \midrule
			\textbf{Non-targeted moments} & Dropout & Highschool & College & Population \\ \midrule
			Percent of total wealth (data) & 0.8 & 17.9 & 81.2 & 100 \\
			Percent of total wealth (model) & 12.4 & 18.6 & 69.0 & 100 \\
			Avg. annual MPC (model, incl. splurge) & 0.79 & 0.78 & 0.54 & 0.69
			\\ \bottomrule 
		\end{tabular}
	\end{table}


	\hyperlink{ConsistentWithMicroData}{\beamerbutton{Go back}}
\end{frame}




\begin{frame}
	\frametitle{Robustness: Different replacement rates}
	\centering  
	\small 
	
	\begin{itemize}
		\item Discount factor distributions: 
	\end{itemize}

	\begin{table}
		\begin{tabular}{llc|cccccc} 
			\toprule
			& & & \multicolumn{2}{c}{Dropout} & \multicolumn{2}{c}{Highschool} & \multicolumn{2}{c}{College} \\ \midrule 
			& & Splurge & $\beta$ & $\nabla$ & $\beta$ & $\nabla$ & $\beta$ & $\nabla$ \\ \midrule 
			Basel. & ($\rho_{b}=0.7$, $\rho_{nb}=0.5$) & 0.307 & 0.694 & 0.542 & 0.904 & 0.099 & 0.978 & 0.015 \\ 
			Alt. & ($\rho_{b}=0.3$,  $\rho_{nb}=0.15$) & 0.307 & 0.599 & 0.687 & 0.852 & 0.159 & 0.968 & 0.028
			\\ \bottomrule 
		\end{tabular}
	
	
	\begin{itemize}
		\item Welfare results: 
	\end{itemize}
		
		\begin{tabular}{@{}lllll@{}}
			\toprule
			&                    & Stimulus check & UI extension & Tax cut \\ \cmidrule(l){1-5} 
			\multirow{2}{*}{no AD effects} 	& Baseline  ($\rho_{b}=0.7$, $\rho_{nb}=0.5$) 		& 0.011          & 0.580        & 0.002   \\
			& Altern.  ($\rho_{b}=0.3$, $\rho_{nb}=0.15$) 	& 0.043          & 1.913        & 0.003   \\ \cmidrule(l){1-5} 
			\multirow{2}{*}{AD effects}		& Baseline  ($\rho_{b}=0.7$, $\rho_{nb}=0.5$)    	& 0.171          & 1.266        & 0.065   \\
			& Altern.  ($\rho_{b}=0.3$, $\rho_{nb}=0.15$)    & 0.169          & 2.620        & 0.052   \\ \cmidrule(l){1-5} 
		\end{tabular}
	\end{table}
\end{frame}

\begin{frame}
	\frametitle{Robustness: Different interest rates}
	\begin{tabular}{lc|cccccc} 
		\toprule
		& & \multicolumn{2}{c}{Dropout} & \multicolumn{2}{c}{Highschool} & \multicolumn{2}{c}{College} \\ \midrule 
		& Splurge & $\beta$ & $\nabla$ & $\beta$ & $\nabla$ & $\beta$ & $\nabla$ \\ \midrule 
		$R = 1.005$ & 0.307 & 0.701 & 0.520 & 0.909 & 0.099 & 0.983 & 0.014 \\
		$R = 1.01$ (baseline) & 0.307 & 0.694 & 0.542 & 0.904 & 0.099 & 0.978 & 0.015 \\ 
		$R = 1.015$ & 0.307 & 0.691 & 0.542 & 0.899 & 0.099 & 0.973 & 0.016 
		\\ \bottomrule 
	\end{tabular}
\end{frame}



\begin{frame}
	\frametitle{Welfare measure construction}
	\hypertarget{WelfareMeasure1}{}
	
	Guiding principles
	
	\begin{enumerate}
		\item Each consumer is valued equally by the social planner 
		\item Utility from splurge in the same way as other spending
		\item No social benefit to the policies outside of a recession
	\end{enumerate} 
	
	\vspace{0.6cm}
	
	Simple aggregation of consumer util. only satisfies principle 1 \& 2:
	\begin{align*}
		\mathcal{W}(\text{policy},Rec,AD) =\frac{1}{N}\sum_{i=1}^{N} \sum_{t=0}^{\infty} \beta_S^t u(\mathbf{c}_{it,\text{policy},Rec,AD}) 
	\end{align*}
	
	\begin{itemize}
		\item $\mathbf{c}_{it,\text{policy},Rec,AD}$: consumption paths (including splurge) for each consumer / policy
		\item $Rec\in\{1,0\}$: recession indicator, $AD\in\{1,0\}$: AD ind.
		\item $\beta_S = 1/R$: social planner's discount factor 
	\end{itemize}	

	\vspace{1cm}
	\hyperlink{WelfareMeasure}{\beamerbutton{Go back}}
	
\end{frame}


\begin{frame}
	\frametitle{Welfare measure construction II}
	\hypertarget{WelfareMeasure2}{}
	
	To satisfy principle 3 we define $\mathcal{C}(\text{policy},Rec,AD) =$
	\begin{align*}
		& \bigg( \underbrace{\frac{\mathcal{W}(\text{policy},Rec,AD)-\mathcal{W}(\text{None},Rec,AD)}{\mathcal{W}^c}}_\text{\RNum{1}}  - \underbrace{\frac{PV(\text{policy},Rec)}{\mathcal{P}^c} }_\text{\RNum{2}} \bigg) \nonumber \\  
		& -
		\bigg( \underbrace{\frac{\mathcal{W}(\text{policy},0,0) - \mathcal{W}(\text{None},0,0)}{\mathcal{W}^c}}_\text{\RNum{3}}  - \underbrace{\frac{PV(\text{policy},0)}{\mathcal{P}^c}}_\text{\RNum{4}}  \bigg) 
	\end{align*}
	
	\begin{itemize}
		\item \RNum{1}: Policy-induced increase in agg. welfare (in bp of SS-cons.)
		\item \RNum{2}: Cost of policy $\Leftrightarrow$ \RNum{1} - \RNum{2}: Net agg. welfare increase
		\item \RNum{3} - \RNum{4}: Net welfare impact of policy outside of recession
		\item $\mathcal{C}$ measures only welfare effects beyond pure redistribution
	\end{itemize}

	\vspace{1cm}
	\hyperlink{WelfareMeasure}{\beamerbutton{Go back}}
	
\end{frame}


\end{document}

